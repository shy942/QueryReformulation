\documentclass[conference]{IEEEtran}
\IEEEoverridecommandlockouts
% The preceding line is only needed to identify funding in the first footnote. If that is unneeded, please comment it out.
\usepackage{cite}
\usepackage{amsmath,amssymb,amsfonts}
\usepackage{algorithmicx}
\usepackage{graphicx}
\usepackage{textcomp}
\usepackage{xcolor}
\usepackage[numbers]{natbib} 
\usepackage{algorithm}
\usepackage{algpseudocode}
\usepackage{multirow}
\usepackage{pifont}
\newcommand{\cmark}{\ding{51}}%
\newcommand{\xmark}{\ding{55}}%
\def\BibTeX{{\rm B\kern-.05em{\sc i\kern-.025em b}\kern-.08em
    T\kern-.1667em\lower.7ex\hbox{E}\kern-.125emX}}
\begin{document}

\title{Improved Bug Localization using Association Map and Information Retrieval (BLuAMIR)\\
%{\footnotesize \textsuperscript{*}Note: Sub-titles are not captured in Xplore and
%/should not be used}
%\thanks{Identify applicable funding agency here. If none, delete this.}
}

\author{\IEEEauthorblockN{Shamima Yeasmin~~~Mohammad Masudur Rahman~~~Chanchal K. Roy~~~ Kevin A. Schneider}
\IEEEauthorblockA{\textit{Department of Computer Science} \\
\textit{ University of Saskatchewan}\\
Saskatoon, Canada \\
shy942@mail.usask.ca, \{masud.rahman, chanchal.roy, kevin.schneider\}@usask.ca}
%\and
%\IEEEauthorblockN{Mohammad Masudur Rahman}
%\IEEEauthorblockA{\textit{dept. name of organization (of Aff.)} \\
%\textit{name of organization (of Aff.)}\\
%City, Country \\
%email address}
%\and
%\IEEEauthorblockN{Chanchal K. roy}
%\IEEEauthorblockA{\textit{dept. name of organization (of Aff.)} \\
%\textit{name of organization (of Aff.)}\\
%City, Country \\
%email address}
%\and
%\IEEEauthorblockN{Kevin A. Schneider}
%\IEEEauthorblockA{\textit{dept. name of organization (of Aff.)} \\
%\textit{name of organization (of Aff.)}\\
%City, Country \\
%email address}

}

\maketitle

\begin{abstract}
Bug localization is one of the most challending tasks undertaken by the developers during software maintenance.
Existing studies mostly rely on lexical similarity between the bug reports and source code for bug localization.
However, such similarity always does not exist, and these studies suffer from vocabulary mismatch issues.
In this paper, we propose a bug localization technique that (1) not only uses lexical similarity between bug report and source code documents  
but also (2) exploits the co-occurrences between keywords from the past reports and source links from corresponding changed code.
Experiments using a collection of ~6000 bug reports show that our technique can locate buggy files with a Top-10 accuracy of 64.05\% and a mean reciprocal rank@10 of 0.31 and a mean precision average@10 of 37\%, which are highly promising. 
Comparison with state-of-the-art IR-based bug localization techniques confirms superiority of our technique. 
\end{abstract}

\begin{IEEEkeywords}
bug localization, bug report, source codes, association map
\end{IEEEkeywords}

\section{Introduction}
Bug localization is the process of locating the source codes that need to be changed in order to fix a given bug. 
Locating buggy files is time-consuming and costly if it is done by manual effort, especially for the large software system when the number of bug of that system becomes large. Therefore, effecting methods for locating bugs automatically from bug reports are highly desirable. 
In automatic bug localization technique, it takes a subject system as an input and produces a list of entities such as classes, methods etc. against a search query. For example, information retrieval based techniques rank the list of entities by predicted relevance and return a ranked list of entities or source codes which may contain the bug. 
The bug localization techniques are also affected by the fact of designing effective query. If a query contains inadequate information, then the retrieval results will not be relevant at all. One other thing that the performance of existing bug localization approach did not reach to an accepted level and so far those studies showed good results for a small set of bugs. Therefore,
%However, static bug localization technique has an advantage over dynamic technique, as it does not require any working subject system rather it can apply on any stage of a system. 
in this paper, we apply a bug localization technique on large dataset that exploits an association link established from bug report repository to source code base through commit logs. 

In existing studies, information retrieval techniques  \cite{Jian} \cite{Saha} \cite{Moreno} \cite{Anh} \cite{Lukins} are applied to automatically search for relevant files based on bug reports.
In one case of Text Retrieval (TR) based approaches, \citet{Jian} propose BugLocator using revised Vector Space Model (rVSM), which requires the information extracted from bug reports and source codes. One of the issue association with TR-based technique is treating source codes as flat text that lacks structure. But exploiting source code structure can be a useful way to improve bug localization accuracy.
Due to the fuzzy nature of information retrieval, textually matching bug reports against source
files may have to concede noise (less relevant but accidentally matched words). However, large files are more
likely to contain noise as usually only a small part
of a large file is relevant to the bug report. So, by treating
files as single units or favoring large files, the techniques are more
likely to be affected by the noise in large files. So, \citet{Saha} present BLUiR, which retrieve structural information from code constructs.
However, bug reports often contain stack-trace information, which may provide direct clues for possible faulty files. Most existing approaches directly treat the bug descriptions as plain texts and do not explicitly consider stack-trace information. Here, \citet{Moreno} combine both stack trace similarity and textual similarity to retrieve potential code elements. 
To deal with two issues associated with source code structure and stack trace information, 
\citet{Chu}  proposed  a technique that use segmentation i.e., divides each source code file into segments and use stack-trace analysis, which significantly improve the performance of BugLocator.

LDA topic-based approaches \cite{Anh} \cite{Lukins} assume that the textual contents of a bug report and it's corresponding buggy source files share some technical aspects of the system.
Therefore, they develop topic model that represents those technical aspects as topic. 
However, existing bug localization approaches applied on small-scale data for evaluation so far.
Besides the problem of small-scale evaluations, the performance of the existing bug localization methods can be further improved too. For example, using Latent Dirichlet Allocation (LDA), only buggy files for 22\% of Eclipse 3.1 bug reports are ranked in the top 10 [25]. 
But, now it is an important research question to know how effective these approaches are for locating bugs in large-scale (i.e., big data).
In the field of query processing
\citet{Sisman}  proposed a method that examines the files retrieved for the initial query supplied by a user and then selects from these files only those additional terms that are in close proximity to the terms in the initial query. Their \cite{Sisman} experimental evaluation on two large software projects using more than 4,000 queries showed that the proposed approach leads to significant improvements for bug localization and outperforms the well-known QR methods
in the literature.
There are two common issues associated with existing studies. First, 
most of the bug localization techniques exploit lexical similarity measure for retrieving relevant files from the code base. So, it is expected that the search query constricted from new bug report should contain keywords similar to code constructs in code base. This issue demands developers or users  previous expertise on the given project, which can not be guaranteed in real world. 
Second, closely related to the first issue there is another well known problem called vocabulary mismatch. In order to convey the same concept on both search query (i.e., new bug report) and source code files the developers tend to use different vocabularies. This issue questions the applicability of exploiting lexical similarity measure.

So, in order to resolve these two issues, in our work we propose a bug localization approach that combines lexical similaority with word co-occurrence measue. 
Our proposed technique exploits the association of keywords extracted from fixed bug report with their changed source code location. Here, the main idea is to capture information from a collection of bug reports and exploit them for locating relevant source code location.
So our approach not (1) only relies on the lexical similarity measure between big reports and source code files for bug localization, and (2) but also addresses the vocabulary mismatch problem by using a keyword-source map constructed from fixed bug information. Moreover, we named our proposed tool as BLuAMIR (Bug Localization using Association Map and Information Retrieval).

We compare the performance of our proposed tool with a state of the art bug localization techniques . So, contributions of our paper include:
\begin{itemize}
 \item A novel bug localization technique that exploits the association relationship between bug report keywords with their buggy source files.
 \item Comprehensive evaluation of the technique using about 3000 bugs from Eclipse bug repository and validation against state-of-the-art.
\end{itemize}


\section{An Example Use Case Scenario}\label{sec:usecase}
\begin{table*}[t]
	\centering
	\caption{A working example of BLuAMIR}
	\label{tab:workingexample}
	\resizebox{7.0in}{!}{%
		\begin{tabular}{c|c|c|c||c|c|c|c|c}
			\hline
			%\begin{tabular}[c]{@{}c@{}}\#Bugs for \\ developing \\ map databases\end{tabular} &
			\textbf{Query \#ID}
			& \textbf{VSM score}  
			& \textbf{Retrieved Files (VSM Approach)}  & 
			\textbf{GoldSet}  &
			\textbf{VSM score}  & 
			\textbf{Co-Occ-Score} &
			\textbf{ScoreAll} &
			\textbf{Retrieved Files (BLuAMIR)} &
			\textbf{Goldset}\\
			\hline
		    \multirow{10}{*}{\#37026} & 1.00 & WorkingSetLabelProvider.java  & \checkmark &0.60  & 0.25  & 0.85 & WorkingSetLabelProvider.java & \checkmark \\
			\cline{2-9}
		    & 0.80 & ImageFactory.java & \xmark & 0.35 & 0.35 & 0.70 & WorkingSet.java & \checkmark \\ 
		  \cline{2-9}
		   & 0.76 & WorkingSetMenuContributionItem.java & \xmark &0.45 &0.22 & 0.67 & IWorkingSet.java & \checkmark\\
		   \cline{2-9}
		   & 0.75 & IWorkingSet.java & \checkmark &0.38 & 0.25 & 0.63 & WorkingSetTypePage.java & \xmark\\
		  \cline{2-9}
		   & 0.70 & ProjectImageRegistry.java & \xmark &0.30 &0.33 & 0.63 & CommandImageService.java & \xmark \\
		   \cline{2-9}
		   & 0.68 & IWorkingSetManagerTest.java & \xmark &0.31 & 0.29 & 0.60 & WorkbenchImages.java & \checkmark\\
		   \cline{2-9}
		   & 0.67 & IWorkbenchPartDescriptor.java & \xmark & 0.33 & 0.27 & 0.60 & WorkingSetAdapterFactory.java & \checkmark \\
		   \cline{2-9}
		   & 0.65 & MockWorkingSetPage.java & \xmark  & 0.30 & 0.27 & 0.57 & EditorIconTest.java & \xmark\\
		 \cline{2-9}
		   & 0.65 & MissingImageDescriptor.java & \xmark  &0.30 &0.27 & 0.57 & PerspectiveDescriptor.java & \xmark \\
		  \cline{2-9}
		   & 0.64 & WorkingSetTypePage.java & \xmark & 0.24 & 0.33 & 0.57 & IAction.java & \xmark\\
		   \hline
	\end{tabular}}
	\centering
\end{table*}
Consider a new bug (ID 37026) is submitted to the Eclipse-UI-Platform bug repository system. The title of this bug is \textit{[Working Sets] IWorkingSet.getImage should be getImageDescriptor}. We preprocess this title as well as the description of this new bug froming a query. Now we apply both VSM and our proposed techniques on this query in order to locate possible buggy files. The recommended results are presented in Table \ref{tab:workingexample}. 
From version repository, we have found that so far 14 number of files have been modified in order to fix this bug. From Table \ref{tab:workingexample}, we can see that our proposed tool BLuAMIR correctly identifies 5 buggy files where VSM approach locates only 2 buggy files. The ranking of BLuAMIR is 1,2,3,6,7 where the rank from VSM is 1 and 4. Note that BLuAMIR retrieves total 5 buggy files including the 2 located by VSM technique. However, VSM and Co-Occ scores are also provided for each recommended buggy file. 
Note that Same file is retrieved in rank 1 for both techniques. The buggy file which is retrieved in rank 4 by VSM, retrieved as rank 3 by BLuAMIR. The other 3 buggy files recommended by BLuAMIR having rank of 12, 27 and 24  in the VSM approach retrieval.

We try to reason why BLuAMIR can locate more buggy files than VSM. In BLuAMIR, the Co-Occ score comes from the association map, which is created from keywords extracted from previously fixed bug report into their fixed buggy files. In VSM, we compare term presented in the bug reports and source files. But we know, there exist a vocabulary missmatch problem between bug reports and source corpus. But, in computing Co-Occ rank we dont directly compare terms between bug reports and source files. So vocabulary missmatch problem is resolved here. Moreover, in the working example, this association map aids locating more buggy files than VSM technique. IF we go deeper, we can find that due to vocabulary missmatch problem, VSM failed to retrieve buggy files where our proposed tool utilize an association map that successfully retrieve previous relationship between bug report keywords and their source buggy files. 

\section{Research Question Formulation}
Our proposed tool-BLuAMIR are designed to  answer the following research questions.
\begin{itemize}
	\item RQ1: How many bugs can be successfully located by BLuAMIR?
	\item RQ2: Does our proposed approach-BLuAMIR resolve the vocabulary missmatch problem?
	\item RQ3: How does BLuAMIR eliminate vocabulary missmatch problem?
	\item RQ4: In BLuAMIR comparable with the state-of-the-art techniques in identifying buggy files?
\end{itemize}

\section{Existing Approaches}\label{sec:existing}
As in our proposed tool, we combine lexical similarity with word co-occurence score, here we discuss TF-IDF based bug localization approach.
\subsection{Lexical Similarity Based Bug Localization Technique:}
In this technique, each source code file is ranked based on source code file scores. Source code file contains words those can be also occurred in the bug reports. This is considered as a hint to locate buggy files. 
%If a new bug is similar to a given previously located bug, then there is a possibility that the source code files located for the past bug can provide useful information in finding buggy files for that new bug.
For locating a new bug we compute similarity scores for all source code files for a given project. However, we need to focus on some concepts which are required to understand our proposed system. They are described as follows:

\textbf{Ranking based on Classical Vector Space Mode:}
The basic idea of a VSM (Vector Space Model) or TF-IDF model is that the weight of a term in a document is increasing with its occurrence frequency in this specific document and decreasing with its occurrence frequency in other documents \cite{Jian}.
In our proposed approach we have used both classical Vector Space Model (VSM) and revised Vector Space Model (rVSM) proposed by \citet{Jian} in order to index and rank source code files. 
In classic VSM, \textit{tf} and \textit{idf} are defined as follows:
\begin{equation}
tf(t,d)=\frac{f_{td}}{\#terms}, idf(t)=log\frac{\#doc}{n_{t}}
\end{equation}
Here \textit{tf} is the term frequency of each unique term \textit{t} in a document \textit{d} and \textit{f\textsubscript{td}} is the number of times term \textit{t} appears in document \textit{d}.
So the equation of classical VSM model is as follows
\begin{multline}\label{VSMequation}
VSMScore(q,d)= cos(q,d) =
\\
\frac{1}{\sqrt{\sum_{t\epsilon q}}((\frac{f_{tq}}{\#terms})\times log(\frac{\#docs}{n_{t}}))^{^{2}}}\times 
\\
\frac{1}{\sqrt{\sum_{t\epsilon d}((\frac{f_{td}}{\#terms})\times log(\frac{\#docs}{n_{t}}))^{2}}}\times
\\
\sum_{t\epsilon q\bigcap d}(\frac{f_{tq}}{\#terms})\times (\frac{f_{td}}{\#terms})\times log(\frac{\#docs}{n_{t}})^{2}
\end{multline}
This \textit{VSM} score is calculated for each query bug report \textit{q} against every document \textit{d} in the corpus. However, in the above equation \textit{\#terms} refers to the total number of terms in a corpus, \textit{n\textsubscript{t}} is the number of documents where term \textit{t} occurs.

\textbf{Ranking based on Revised Vector Space Mode:}
The main difference between classic VSM and revised VSM is that in case of revised version logarithm variant is used in computing term frequency. The equation for calculating term frequency is:
\begin{equation}
tf(t,d)=log(f_{td})+1
\end{equation}
So the new equation of revised VSM model is as follows:
\begin{multline}\label{rVSMequation}
rVSMScore(q,d)=g(\#term)\times cos(q,d)
\\
\frac{1}{1+e^{-N(\#terms))}}\times \frac{1}{\sqrt{\sum_{t\epsilon q}}((logf_{tq}+1)\times log(\frac{\#docs}{n_{t}}))^{^{2}}}\times 
\\
\frac{1}{\sqrt{\sum_{t\epsilon d}((log {f_{td}+1})\times log(\frac{\#docs}{n_{t}}))^{2}}}\times
\\
\sum_{t\epsilon q\bigcap d}(logf_{tq}+1)\times (logf_{td}+1)\times log(\frac{\#docs}{n_{t}})^{2}
\end{multline}
%\end{equation}
This \textit{rVSM} score is calculated for each query bug report \textit{q} against every document \textit{d} in the corpus. However, in the above equation \textit{\#terms} refers to the total number of terms in a corpus, \textit{n\textsubscript{t}} is the number of documents where term \textit{t} occurs.

\textbf{Ranking based on similar bug information}
\begin{figure}
	\centering
	\includegraphics[scale=0.65]{3layers}
	\caption{Bug and its Similar Bug Relationship with Source Code Files:}
	\label{fig:BSBR}
\end{figure}
The assumption of this ranking is similar bugs of a given bug tend to modify similar source code files. Here, we construct a 3-layer architecture as described in \cite{Jian}. In the top layer (layer 1) there is a bug \textit{B} which represents a newly reported bug. All previously fixed bug reports which have non-negative similarity with bug  \textit{B} are presented in second layer. In third layer all source code files are shown. In order to resolve each bug in second layer some files in the corpus were modified or changed, which are indicated by a link between layer 2 and layer 3. Foe all source code files in layer 3, similarity score is computed, which can be referred to as degree of similarity. The score can be defined as:

\begin{equation}\label{Simiequation}
SimiScore=\sum_{All S_{i} that connect to F_{j}}(Similarity(B,S_{i})/n_{i})
\end{equation} 
Here, similarity between newly reported bug \textit{B} and previously fixed bug \textit{S\textsubscript{i}} is calculated based on cosine similarity measure and \textit{n\textsubscript{i}} is the total number of link \textit{S\textsubscript{i}} has with source code files in layer 3.

%%\subsection
%\textbf{Combining Both Ranks:}
%We combine the both scores based on source code score and similar bugs score as in \cite{Jian} as follows:
%\begin{equation}
%FinalScore=(1-\alpha )\times N(rVSMScore)+\alpha \times N(SimiScore)
%\end{equation}



%\subsection{LDA Topic Based Bug Localization Technique:}
%The main assumption behind this technique is the textual content of the bug reports and their associated buggy source code files tend to describe some common technical aspects. So, if  we could identify the technical topics extracted from both bug reports and source codes, we could recommend the files shared technical topics with the newly reported bug reports. If a newly reported bug has similar technical topics with some previously fixed bug reports, the fixed files could be a good candidate files for newly reported bug.
%
%LDA (Latent Dirichlet Allocation) is a probabilistic and fully generative topic model. It is used to extract latent (i.e., hidden) topics, which are presented in a collection of documents. It also model each document as a finite mixture over the set of topics [add link here]. In LDA, similarity between a document \textit{d} and a query \textit{q} is computed as the conditional probability of the query given that document \cite{Lukins2}.
%
%\begin{equation}
%Sim(q,d_{i})=P(q|d_{i})=\prod_{q_{k}\epsilon q}P(q_{k}|d_{i})
%\end{equation}
%
%Here \textit{q\textsubscript{k}} is the \textit{k}th word in the query {q} Thus, a document (i.e., source code file) is relevant to a query if it has a high probability of generating the words in the query.
%
%We perform the following steps in order to localize buggy files for newly reported bug using LDA based approach:
%
%\begin{itemize}
%	\item Apply topic modeling on the source code files. The output contains a certain number of topics and some associated keywords for each topic. We also get some other distribution files such as document-topic, word-topic etc.
%	\item Now work with the documents topic distribution file. Make a list of source code documents or files for each topic. So, we wiill have a list that contain all topics and their associated source code documents.
%	\item Here our query is the newly reported bug. This contains information in the bug reports such as title and short description etc. We all do inference for this query using a topic modeling tool. It will extract all topic associated with the query (i.e., newly reported bug).
%	\item Now we need to work with topic keywords. We are going to perform a comparison between newly reported bug or the given query and source code files using topic information. That means we will compare topic-keywords associated with topics inferred for the query with topic-keywords of each topic extracted from source code documents.
%	\item We will rank them based on topic-keyword similarity. So, now we know which are the top most topics, and we already have information regarding topic-document relationship, we will retrieve all source code files associated with all those top most topic as recommended buggy files.
%	%\item A detailed description of methodologies for visualizing topic evolution extracted from bug reports.
%	%\item A detailed description of methodologies for visualizing bug report extractive summaries.
%	%\item Evaluation of visualized bug report extractive summary by conducting a task-oriented user study.
%\end{itemize}
\section{Proposed System Diagram}\label{sec:proposedsystemDiagram}
Our proposed approach combine lexical similarity and co-occurence similarity measure.  \citet{Jian} proposed BugLocator based on two different similarity scores- one is rVSM score and the other one is Simi score. 
In BLuAMIR, we follow two different approach for computing source code ranks. We retrieve relevant ranked source files based on 1) three different scores- i) rVSM, ii) Simi and iii) Word Co-occurence and 2) two different scores - i) VSM and ii) Word Co-occurence.  So,
we have divided our approach into two different sections or parts- 1) calculate rVSM and Simi scores or VSM score and co-occurence measure and 2) combine all three or two scores in order to localize recommended buggy source files for a given newly reported bug. As we have combined two existing scores with our proposed word co-occirence score, we will discuss the system diagram of association map database in Part I and then we represent rest of our system diagram in Part II.

\subsection{Part I: Mapping Bug Source Code Links}
In Part I, we construct an association map database - between keywords extracted from previously fixed bug reports collection and source code links. The system diagram for Part I is given in Fig. \ref{fig:systemDiagram}(a).

This mapping construction involves two steps.
At first we create association map using information contained in bug reports collection and commit logs. We collect information from title and description fields of each bug report. There are also developers comment section in a bug report, which is highly informative. But large documents also tend to reduce performance by taking too long to process. So, we reluctant to include developers comments of each bug report. However, we extract keywords from each bug report after preprocessing them such as stop word removal. 
%In this part, information are collected from bug report, version repository and source code repository. 
In version repository, we have commit logs, where the developers commit for several changes in a software project. For example, when a bug report is fixed, the developer who fixed this bug also creates a commit log containing which type of change he had to made to resolve the bug associated with the location of the source code files where the change has made. So, if we analyze commit logs, we can retrieve source code files information which have been changed in order to fix a given bug. Here the idea is, we first extract keywords from bug report and source code links for the same bug from commit log and then construct an association map database containing this mapping information. The links between keywords and source code links can be described as: each or several keywords can be linked to one or more source code links and each source code link can be linked to one or several keywords. This way we connect keywords and source code links to form an association map database named keyword-source code links. 




\begin{figure*}
	\centering
	\includegraphics[scale=0.65]{SD3}
	\caption{Proposed System Diagram: (a) Mapping Bug Source Code Links and  (b) Bug Localization Using VSM, and Co-occurence Ranks}
	\label{fig:systemDiagram}
\end{figure*}

\subsection{Part II: Bug Localization Using either rVSM and Simi or VSM, and Co-occurence Ranks}

The system diagram for this part has illustrated in Fig. \ref{fig:systemDiagram}(b). In BLuAMIR, we calculate similarity score in two ways - one is based on rVSM, Simi and Co-occurence scores and other one is the combination of VSM and Co-occurence scores. However, for simplicity we represent the later approach in our proposed schematic diagram in Fig. \ref{fig:systemDiagram}(b).
In the first approach, we have three different ranks i.e., rVSM, Simi and Co-occurence. We utlize an existing technique proposed by \citet{Jian} for computing rVSM and Similarity scores. Basically, rVSM is a TF-IDF based score, which is measured between query and source code files.
On the other hand, Simi score refers to that fact that if a bug is similar to another bug, then they both tend to relate to same sources. However, we describe both scores in Section \ref{sec:existing}.
We have constructed an association map databases from keywords collected from bug report to its source code location information in Section \ref{sec:proposedsystemDiagram}.
For the candidate keyword tokens for the initial query, we exploit association map database (i.e., keyword-source code links) and retrieve the relevant source code files. We use some heuristic functions in order to combine three ranks and recommend buggy relevant source files. 


\section{Proposed Approach}\label{sec:proposedApproach}
Our proposed approach consists of two parts - (i) constructing association map databases and (ii) retrieve relevant buggy source code files. 

\subsection{Construction of Association Map Database Between Bug Reports and Source Files}
In this part, we construct an association map databases - between keywords and source code links extracted from bug reports and commit messages respectively. This section can be further divided into several parts: keyword extraction from bug reports, source code link extraction from commit logs, keyword- source code linking. They are discussed in the followings:

\begin{algorithm}[!t]
	\caption{Construction of Association Map Database Between Bug Reports and Source Files}
	\label{map}
	\begin{algorithmic}[1]
		\Procedure{BUG REPORTS }{$BRC$}\Comment{$BRC$: a collection of bug reports}
		\State $MAP_{KS} \gets$ \{\}\Comment{an association map}
		\Comment{creating adjacency map database from the bug reports collection}
		\State $MAP_{adj} \gets$ createAdjacencyDatabase($BRC$)
		\Comment{creating a map that links keywords into their bug ids}
		\State $MAP_{kb} \gets$ createKeywordsToBugMaps($MAP_{adj}$)
		\Comment{collecting unique keywords from keywords to bug map}
		
		%\LineComment{preprocess the collected keywords}
		%\For{Keyword $K_i \in$ $K$}
		\State $KB \gets$ collectKeywords($MAP_{kb}$)
		
		\Comment{Linking keywords from a bug report into its change set files}
		\For{keywords $KB_{_i}$ $\in$ $KB$}
		\State $BUG_{id} \gets$ retrieveBugIds ($K_i$)
		\For{each bug id $BUG_{id_j} \in BUG_{id}$}
		\State $SF_{links} \gets$ getLinkedSourceFiles($BUG_{id}$)
		\Comment{maps all source code files to its keywords}
		%\State $MAO_{BUG_{id}} \gets$ linkSourceCodeFiles$Adj_{T_i}, Adj_{K_j}$) 
		\State $MAP[KB_i].links \gets MAP[KB_{i}].links + SF_{links}$
		\EndFor
		\EndFor
		
		\Comment{collecting all keyword-source files links}
		\State $MAP_{KS} \gets$ $MAP[KB]$ 
		%\LineComment{put them into map}
		%\State $MAP \gets$ mapping()
		\State \textbf{return} $MAP_{KS}$
		\EndProcedure
	\end{algorithmic}
\end{algorithm}



\textbf{Keyword Extraction from Bug Reports:} We collect title, description and developers comments from each fixed bug report in a collection of bug reports. We perform standard natural language pre-processing on the corpus such as stop word removal, splitting and stemming. The purpose of removing stop words is that they contain very little semantic for a sentence. The process of stemming step extracts the root of each of the word. We use this stop word list [link] during stop word removal and Porter Stemming stemmer \footnote{{1}http://tartarus.org/martin/PorterStemmer/} for stemming. 

%{\footnotesize \textsuperscript{1}http://tartarus.org/martin/PorterStemmer/}


\textbf{Source Code Link Extraction from Commit Logs:}
We go through all commit messages and try to find those commit message that contain keywords related to bug fix such as fix, fixed and so on. Each of these commit messages presents other information such as the ID of bug report for which it was created and the links of the changed source code files. We then construct a linking relationship against each fixed bug report ID into their changed source code files.

\textbf{Keyword- Source Code Linking:}
At this point, in one side we have pre-processed keywords associated with each bug report and on the other side we have a relationship information between bug report ID and buggy source code links. We construct a bipartite graph between keywords collected from a bug report to its buggy source code locations. Here, one or more keywords can be linked to one or more buggy source code files links and a source code file can be linked to one or more keywords.

\textbf{Calculating Co-occurence Scores}
A query typically contains several keywords or words. For each keyword, we look for relevant source files in the keyword-source files association map. We assume these files are relevant because we created the map between the content of bug reports and their buggy source files. When we analysis these links for all keywords in a query, a relevant file can be found from the association relationship more than once. So, them we normalize the frequency of source files using standard TFIDF normalization technique. Then we recommend first Top-K files with their CoOccScores. The equation for computing co-occurence score is given belows:
\begin{equation}\label{CoOccequation}
CoOccScore=\sum_{All S_{i} that connect to W_{j}}(Link(W{j},S_{i}))
\end{equation}
Here, the link $Link(W{j},S_{i})$ between keyword and source file is 1 if they are connected in the association map and 0 otherwise.


\subsection{Localizing Buggy Source Code Files} 
In this part, we combine existing two lexical similarity based bug localization approaches with our proposed keyword-source co-occurence relations. One is with rVSM and Simi ranks which are presented in BugLocator and other one is Vector Space Model based technique. So, there are two sub parts of this section.

\textbf{Combination of rVSM, Simi Rank and Co-occurence Rank:}

For each query, we compute the rVSM score against all source codes in the database using equation \ref{rVSMequation} and we also calculate Simi score using equation \ref{Simiequation}. Then we calculate co-occurence scores for the query using equation \ref{CoOccequation}.
We finally combine the three ranks and for that we use three weighting factor {$\alpha$}, $\beta$ and $\gamma$.
The final equation is given in equation \ref{equationBLme}.
\begin{multline}\label{equationBLme}
FinalScoreApproach1=\gamma \times N(rVSMScore)+
\\ \beta \times N(SimiScore) + \alpha \times N(CoOccScore)
\end{multline}
We work with different values of $\alpha$, which are presented in the experiment section. We use value of 0.2 for $\beta$, varying $\alpha$ from 0.1 to 0.5 and thus, $\gamma$ from 0.7 to 0.3. So that they end up into 1.

\textbf{Combinition of VSM and Co-occurence Rank:}

We compute VSM score using Appache Lucene library. Then we combine that score with our CoOccScore using equation \ref{equationVSMme}.
\begin{multline}\label{equationVSMme}
FinalScoreApproach2=(1-\alpha )\times N(VSMScore)+ \\
\alpha \times N(CoOccScore)
\end{multline}
Here, the weighting factor $\alpha$ varies from 0.1 to 0.5, for which we discuss results in the experiment section.

\section{Experiment and Discussion}
In this section, at first we discuss detail of our data set, then we describe the evaluation metrics and finally we present our experimental results.



%\begin{table}[htbp]
%\caption{Table Type Styles}
%\begin{center}
%\begin{tabular}{|c|c|c|c|}
%\hline
%\textbf{Table}&\multicolumn{3}{|c|}{\textbf{Table Column Head}} \\
%\cline{2-4} 
%\textbf{Head} & \textbf{\textit{Table column subhead}}& \textbf{\textit{Subhead}}& \textbf{\textit{Subhead}} \\
%\hline
%copy& More table copy$^{\mathrm{a}}$& &  \\
%\hline
%\multicolumn{4}{l}{$^{\mathrm{a}}$Sample of a Table footnote.}
%\end{tabular}
%\label{tab1}
%\end{center}
%\end{table}



\subsection{Experimental Dataset}
We work with three different dataset - Eclipse, SWT aand Zxing.
We work with Eclipse data set which is a popular IDE for Java. We downloaed a git based Eclipse project from git repository \footnote{https://git.eclipse.org/c/platform/eclipse.platform.ui.git/}. We work with Eclipse Platform UI project. 
%These source codes are contained in our source code repository. 
On the other hand, currently Eclipse Platform UI project contains more than 10K number of bugs where we only work with the bugs which are fixed. We create quires from each bugs considering their title and short summary.
%We have two parts in our corpus. One is source code files downloaded as git based project and another part is bug reports collection. 
All bug reports are collected from \footnote{https://bugs.eclipse.org/}. In order to obtain the links between previously fixed bugs and source code files, we analyze git project commit message. We ran through all commit messages and track Bug IDs associated with examined source code files. 
In order to evaluate our proposed tool we have also used two more dataset that \citet{Jian} used
to evaluate BugLocator. This dataset contains 118 bug reports in total from two popular open source projects– SWT, and ZXing along with the information of fixed files for those bugs. The detail of our dataset is presented in \ref{tab:DDSl}. SWT is a component of Eclipse \footnote{http://www.eclipse.org/swt/}. Zxing is an android based project maintained by google \footnote{http://code.google.com/p/zxing/}.
\begin{table}[htbp]
	\caption{Description of Data Sets}
	\label{tab:DDSl}
	%\centering
	\begin{center}
		\begin{tabular}{ p{1cm} | p{2.5cm} | p{1.5cm} | p{.75cm} | p{.75cm} }
			\hline
			\textbf{Project Name}  & \textbf{Description} & \textbf{Study Period}& \textbf{\#Fixed Bugs} & \textbf{\#Source Files}\\
			\hline
			{Eclipse Platform Ant} & Popular IDE for Java & & {3000} & 11732\\ \hline
			SWT (V 3.1)& An open source widget toolkit for Java & Oct 2004 - Apr 2010 & 98 & 484 \\ \hline
			Zxing & A barcode image processing library for Android Application & Mar 2010 - Sep 2010 & 20 & 391 \\
			\hline
			
		\end{tabular}
	\end{center}
	%\centering
\end{table}


\subsection{Evaluation Metrices}
To measure the effectiveness of the proposed bug localization approach, we use the following metrics:

\textbf{Ton N-Rank (Hit@N):} It represents the number of bug, for which their associated files are returned in a ranked list. Here, \textit{N} may be 1, 5 or 10. We assume that if at least one associated file is presented in the resulted ranked list, then the given bug is located. The higher the metric value, the better the bug localization performance

\textbf{MRR(Mean Reciprocal Rank)}
The reciprocal rank of a query is the multiplicative inverse of the rank of the first correct answer. So mean reciprocal rank is the average of the reciprocal ranks of results of a set of queries $Q$
\begin{equation}
MRR=\frac{1}{\left | Q \right |}\sum_{i=1}^{\left | Q \right |}\frac{1}{rank_{i}}
\end{equation}
where $ranki$ is the position of the first buggy file in the returned ranked files for the first query in $Q$.

\textbf{MAP(Mean Average Precision)}
Mean Average Precision is the most commonly used IR metric to evaluate ranking approaches. It considers the ranks of all buggy files into consideration. So, MAP emphasizes all of the buggy files instead of only the first one.
MAP for a set of queries is the mean of the average precision scores for each query. The average precision of a single query is computed as:
\begin{equation}
AP=\sum_{k=1}^{M}\frac{P(k)\times pos(k)}{number \, of \, positive \, instances}
\end{equation}
where $k$ is a rank in the returned ranked files, $M$ is the number of ranked files and $pos(k)$ indicates whether the kth file is a buggy file or not. $P(k)$ is the precision at a given top $k$ files and is computed as follows:
\begin{equation}
P(k)=\frac{\#buggy \, Files}{k}
\end{equation}

\textbf{Wilcoxon signed-rank test}
The Wilcoxon signed-rank test is a non-parametric statistical hypothesis test used to compare two related samples, matched samples, or repeated measurements on a single sample to assess whether their population mean ranks differ. We perform this test with the help of \footnote{https://www.socscistatistics.com/tests/signedranks/Default.aspx}.

\textbf{Cross Validation}
We divide our query data into k number of sets. Typically k is 10, but we work with k =5 and k=10. Each set contains a training set and tesing set. Training data is used to create mapping between keywords extracted from bug reports and source code files. 10-fold-cross validation data is presented in table \ref{tab:Performance1} and table \ref{tab:Performance2}.
\begin{table}[htbp]
\caption{Performance of Bugloactor and proposed technique (rVSM+Simi+Co-Occerence)}
\label{tab:Performance1}
\centering
\resizebox{3.4in}{!}{%
\begin{tabular}{c|c|c|c|c|c|c}
	\hline
			%\begin{tabular}[c]{@{}c@{}}\#Bugs for \\ developing \\ map databases\end{tabular} &
			\begin{tabular}[c]{@{}c@{}}\# \textbf{Test}  \\ \textbf{Case} \\ \end{tabular} & \begin{tabular}[c]{@{}c@{}}\textbf{Methodology} \\ \end{tabular} &
			%\textbf{alpha} &
			%\textbf{beta}&
			%\textbf{gamma}& 
			\begin{tabular}[c]{@{}c@{}}\textbf{Top 1}\\ \%\end{tabular} & 
			\begin{tabular}[c]{@{}c@{}}\textbf{Top 5}\\ \%\end{tabular} & 
			\begin{tabular}[c]{@{}c@{}}\textbf{Top 10}\\ \%\end{tabular} &
			
			\begin{tabular}[c]{@{}c@{}} \textbf{MRR} \end{tabular} & 
			\begin{tabular}[c]{@{}c@{}} \textbf{MAP} \end{tabular} \\ \hline \hline
			\multirow{2}{*}{1}  &BugLocator     &  8.88& 24.56&34.32& 0.16 & 0.15 \\  \cline{2-7}
			& rVSM+Simi+ Co-Score                                                                                                                                               & 16.27                                               & 39.35                                            & 49.41                                                &   0.26  & 0.25    \\ \hline
			\multirow{2}{*}{2}  &BugLocator     &  9.9& 24.62&34.53& 0.17 & 0.16 \\ \cline{2-7}
			& rVSM+Simi+ Co-Score                                                                      & 18.32                                               & 41.44                                              & 53.45                                             &   0.28  &   0.27  \\ \hline
			\multirow{2}{*}{3}  &BugLocator     &  7.50& 21.32&30.63& 0.14 & 0.13 \\ \cline{2-7}
			& rVSM+Simi+ Co-Score                                                                     & 18.62                                            & 42.34                                            & 52.25                                             &   0.28  &  0.27   \\  \hline
			\multirow{2}{*}{4}  &BugLocator     &  8.1& 21.02&29.13& 0.14 & 0.13 \\  \cline{2-7}
			& rVSM+Simi+ Co-Score                                                                    & 14.71                       & 42.04                       & 54.35                                               &  0.26   &  0.25  \\  \hline
			\multirow{2}{*}{5}  &BugLocator     &  9.6& 30.63&42.94& 0.18 & 0.18 \\  \cline{2-7}
			& rVSM+Simi+ Co-Score                                                                    & 22.22                                                 & 42.34                                                & 57.66                                                  &  0.31   & 0.30     \\ \hline
			\multirow{2}{*}{6}  &BugLocator     &  10.21& 31.53&43.54& 0.19 & 0.19 \\  \cline{2-7}
			& rVSM+Simi+ Co-Score
			&25.53 &
			52.25 &
			60.66 & 0.36 &
			0.34     \\  \hline
			\multirow{2}{*}{7}  &BugLocator     &  9.61& 30.33&40.84& 0.18 & 0.17 \\ \cline{2-7}
			& rVSM+Simi+ Co-Score
			&25.22 &
			52.25 &
			64.56 & 0.36 &
			0.34     \\  \hline
			\multirow{2}{*}{8}  &BugLocator     &  8.4& 26.13&39.94& 0.19 & 0.16 \\ \cline{2-7}
			& rVSM+Simi+ Co-Score
			&24.92 &
			51.95 &
			62.46 & 0.36 &
			0.34    \\  \hline
			\multirow{2}{*}{9}  &BugLocator     &  11.11& 28.83&40.24& 0.19 & 0.18 \\  \cline{2-7}
			& rVSM+Simi+ Co-Score
			&21.92 &
			48.65 &
			62.46 & 0.33 &
			0.32    \\  \hline
			\multirow{2}{*}{10}  &BugLocator     &  6.6& 20.72&27.93& 0.12 & 0.12 \\  \cline{2-7}
			& rVSM+Simi+ Co-Score
			&16.21 &
			40.54 &
			55.55 & 0.27 &
			0.26    \\  \hline \hline
			\multirow{2}{*}{Average}       &BugLocator     &  8.99\%& 25.87\%&36.40\%& 0.16 & 0.16  \\ \cline{2-7}
			& rVSM+Simi+ Co-Score                                                                                                                         & 20.39\%                                                 & 45.32\%                                                 & 57.28\%                                                  &   0.31  &  0.29    \\ 
			\hline
\end{tabular}}
\centering
\end{table}








\subsection{Experimental Results}
During experiment, we evaluate our proposed approach in different ways. To create mapping between bug report keywords and source files, we consider three different options - (1) including only title or summary of a bug report in creating corpus, (2) in addition with title we also include description field of a bug report and (3) full content of a bug report could be an option. Neither option 1 nor 3 provides better result and option 2 optimized the performance. We explain this in a way that providing only title of a big report conveys very little information. On the other hand, including full content of a bug report also create too much information that contains huge noise data and also takes longer time during mapping them into source code files. Therefore, title and description of a bug report optimized those two options. However, considering title and description did not get rid of noise and therefore we discard all keywords that happen to exist in 25\% or more documents in the corpus.

However, we divide our experimental result into two parts. We implemented BLuAMIR by two different approaches. In   part I we work with Eclipse dataset, where we compare the performance of BLuAMIR with BugLocator \cite{Jian} and VSM based bug localization techniques. On the other hand, in part II we experiment with SWT and Zxing dataset as in \cite{Jian}. We also answer our research questions throughout the two parts.


We compare our proposed bug localization approach with two existing techniques - 1) BugLocaotor \cite{Jian} which is based on rVSM and Simi scores and 2) VSM which is based on vector space model. In the following two subsections we will describe the performance comparison between these two existing approaches and our proposed approaches
\begin{table}[htbp]
	\centering
	\caption{Performance of proposed technique (VSM+Co-Occerence) Ranks}
	\label{tab:Performance2}
	\resizebox{3.5in}{!}{%
		\begin{tabular}{c|c|c|c|c|c|c}
			\hline
			%\begin{tabular}[c]{@{}c@{}}\#Bugs for \\ developing \\ map databases\end{tabular} &
			
			\begin{tabular}[c]{@{}c@{}}\# \textbf{Test}  \\ \textbf{Case} \\ \end{tabular} & 
			\begin{tabular}[c]{@{}c@{}}\#\textbf{Methodology} \\  \end{tabular} 
			& 
			\begin{tabular}[c]{@{}c@{}}\textbf{Top 1}\\ \%\end{tabular} & \begin{tabular}[c]{@{}c@{}}\textbf{Top 5}\\ \%\end{tabular} & \begin{tabular}[c]{@{}c@{}}\textbf{Top 10}\\ \%\end{tabular} & 
			\textbf{MRR@10} 
			& \textbf{MAP@10} \\ \hline \hline
			\multirow{2}{*}{1}& VSM & 23.67& 46.59&56.97& 0.33 & 0.32 \\ \cline{2-7}
			& VSM + Co-Score & 28.40                                               & 51.77                                            & 60.06                                                &   0.38  & 0.36 \\ \hline
			\multirow{2}{*}{2}                                                                               & VSM & 24.62 & 48.05 & 57.96 & 0.34 & 0.34 \\  \cline{2-7}  &VSM + Co-Score                                                                     & 27.63                                               & 53.15                                              & 61.26                                             &   0.38  &   0.37  \\ \hline
			\multirow{2}{*}{3}                                                                               & VSM & 20.78 & 40.96 & 51.20 & 0.30 & 0.29 \\  \cline{2-7}   &VSM + Co-Score                                                                       & 23.12                                            & 46.85                                            & 59.76                                             &   0.34  &  0.32   \\ \hline
			\multirow{2}{*}{4}                                                                               & VSM & 22.52 & 42.94 & 53.75 & 0.32 & 0.31 \\   \cline{2-7} &VSM + Co-Score                                                                      & 23.42                       & 49.25                       & 61.56                                                &  0.35   &  0.33  \\  \hline
			\multirow{2}{*}{5}                                                                               & VSM & 27.33 & 52.25 & 63.36 & 0.38 & 0.35 \\   \cline{2-7} &VSM + Co-Score                                                                       & 29.73                                                 & 53.15                                                 & 62.16                                                  &  0.39   & 0.36     \\  \hline
			\multirow{2}{*}{6}                                                                               & VSM & 25.00 & 50.60 & 60.84 & 0.36 & 0.34 \\  \cline{2-7}  &VSM + Co-Score 
			&26.13 &
			51.65 &
			62.76 & 0.37 &
			0.35     \\  \hline 
			\multirow{2}{*}{7}                                                                               & VSM & 29.82 & 54.52 & 66.27 & 0.40 & 0.38 \\  \cline{2-7}  &VSM + Co-Score 
			
			&35.43 &
			60.96 &
			72.07 & 0.46 &
			0.44     \\  \hline
			\multirow{2}{*}{8}                                                                               & VSM & 27.79 & 51.36 & 60.73 & 0.38 & 0.36 \\  \cline{2-7}  &VSM + Co-Score 
			&36.64 &
			58.86 &
			67.87 & 0.46 &
			0.43    \\  \hline
			\multirow{2}{*}{9}                                                                               & VSM & 29.13 & 52.55 & 64.86 & 0.39 & 0.36 \\  \cline{2-7}   &VSM + Co-Score 
			&29.13 &
			61.86 &
			69.97 & 0.42 &
			0.40    \\  \hline
			\multirow{2}{*}{10}                                                                               & VSM & 19.03 & 39.58 & 51.34 & 0.28 & 0.26 \\  \cline{2-7}  &VSM + Co-Score
			&24.62 &
			51.05 &
			63.06 & 0.36 &
			0.34    \\ \hline \hline
			\multirow{2}{*}{Average}                                                                               & VSM & 24.98\% & 47.94\% & 58.73\% & 0.35 & 0.33 \\  \cline{2-7}   &VSM + Co-Score      & 28.43\%                                                 & 53.86\%                                                 & 64.05\%                                                  &   0.39  &  0.37    \\ 
			\hline
	\end{tabular}}
	\centering
\end{table}

\textbf{BugLocator VS Our Proposed Tool:}
We combine rVSM and simi ranks with our co-occurence rank. Here, co-occurence rank is computed based on keyword-source code mapping database.In table~\ref{tab:Performance1} we, compare the performance of our proposed approach in terms of top 1, 5, 10 rank, MRR and MAP. We can see that our proposed approach outperforms in all cases. For example, our Top-10 performance 52.94\% has an improvement than BugLocator (36.40\%).

We also compute Wilcoxon signed-rank test both for MRR and MAP. For MRR the {Z} -value is -2.8031. The {p} -value is 0.00512. The result is significant at p<=0.05. The W-value is 0. The critical value of W for N = 10 at p<=0.05 is 8. Therefore, the result is significant at p<=0.05.
For MAP - the {Z} -value is -2.8031. The {p} -value is 0.00512. The result is significant at p<=0.05. The W-value is 0. The critical value of W for N = 10 at p<=0.05 is 8. Therefore, the result is significant at p<=0.05.


\textbf{VSM vs Our proposed Tool}
We combine VSM score and co-occurence scores in order to produce ranked result. We also compare the performance of baseline VSM technique and our proposed combined approach. The comparison is presented in table \ref{tab:Performance2}. We compute Top-1, Top-5, Top-10 performance and MRR and MAP for both approaches. In all cases our proposed approach outperforms VSM-based bug localization approach. The Top-10 performance of our tool is 64.05\% whereas it is 58.73\% for VSM.

We also compute Wilcoxon signed-rank test both for MRR and MAP. For MRR the {Z} -value is -2.8031. The {p} -value is 0.00512. The result is significant at p<=0.05. The W-value is 0. The critical value of W for N = 10 at p<=0.05 is 8. Therefore, the result is significant at p<=0.05.
For MAP - the {Z} -value is -2.8031. The {p} -value is 0.00512. The result is significant at p<=0.05. The W-value is 0. The critical value of W for N = 10 at p<=0.05 is 8. Therefore, the result is significant at p<=0.05.
\subsection{Comparison with State-of-art Technique}
We compare the performance of BLuAMIR with BugLocator for the the same dataset for SWT and Zxing presented in Table \ref{tab:performance3}. Here, the results from buglocator is directly copied from their paper \cite{Jian}. We also collect the same bug reports and the same source code repository for both of them. So the results can be compared. For SWT, we can see our tool BLuAMIR performs better for Top-1, Top-5, MRR and MAP. However, for Top-10 BLuAmir is comparable with Buglocator. On the other hand, for Zxing our tool BLuAMIR outperforms for top-5, top-10 and MAP. 
\begin{table}[htbp]
	\centering
	\caption{Performance Comparison between BugLocator and BLuAMIR}
	\label{tab:performance3}
	\resizebox{3.6in}{!}{%
		\begin{tabular}{c|c|c|c|c|c|c}
			\hline
			%\begin{tabular}[c]{@{}c@{}}\#Bugs for \\ developing \\ map databases\end{tabular} &
			
			\begin{tabular}[c]{@{}c@{}}\# \textbf{System}   \\ \end{tabular} & 
			\begin{tabular}[c]{@{}c@{}}\#\textbf{Localization} \\ \textbf{Approach} \\ \end{tabular} 
			& 
			\begin{tabular}[c]{@{}c@{}}\textbf{Top 1}\\ \%\end{tabular} & \begin{tabular}[c]{@{}c@{}}\textbf{Top 5}\\ \%\end{tabular} & \begin{tabular}[c]{@{}c@{}}\textbf{Top 10}\\ \%\end{tabular} & 
			\textbf{MRR@10} 
			& \textbf{MAP@10} \\ \hline \hline
			
			\multirow{2}{*}{SWT}                                                                               & BugLocator & 39.80 & 67.35 & 81.63 & 0.53 & 0.45 \\  \cline{2-7}  &BLuAMIR                                                                     & 44.90                                               & 73.45                                              & 80.61                                             &   0.57  &   0.54  \\ \hline
			\multirow{2}{*}{Zxing}                                                                               & BugLocator & 40.00 & 60.00 & 70.00 & 0.50 & 0.44 \\  \cline{2-7}   &BLuAMIR                                                                  & 30.00                                            & 70.00                                           & 75.00                                             &   0.48  &  0.46   \\ \hline
			
			
			
			\hline
	\end{tabular}}
	\centering
\end{table}

\textbf{The Ranking Comparison Between Buglocator and BLuAMIR for SWT dataset}
We also perform a query wise comparison for SWT, which is presented in Fig (link).

\subsection{Weighting Factor Analysis}
For both of our proposed approaches we have used some weighting functions, which are described as follows:

\textbf{Weighting Function for rVSM+Simi+Co-ocuurence Ranking:}
We compute performance TopK accuracy, MRR and MAP for different weighting function such as  $\aleph$ is 0.2, 0.3, 0.4, $\beta$ is 0.2 nad $\gamma$ is 0.6, 0.5, 0.4 respectively. The results are presented in Table \ref{tab:alphaApproach2}. 
 Here, it shows, $\alpha$ produces better performance. That means if we increase the co-occurence scores with higher weighting function, the better performance is resulted. This also prove our co-occurence rank is effective in producing better results. We also represent the impact of $\alpha$ for Top-1, Top-5 and Top-10 retrieval on Eclipse dataset for approach 1 (rVSM+Simi+Co-ocuurence Ranking) in figure \ref{fig:PerformanceTopK}.
\begin{table}[htbp]
\centering
\caption{Performance of (rVSM+Simi+Co-Occerence) for different weighting factors}
\label{tab:alphaApproach1}
\resizebox{3.4in}{!}{%
\begin{tabular}{c|c|c|c|c|c|c|c}
			\hline
			%\begin{tabular}[c]{@{}c@{}}\#Bugs for \\ developing \\ map databases\end{tabular} &
%			\begin{tabular}[c]{@{}c@{}} \textit{alpha} \end{tabular} & \begin{tabular}[c]{@{}c@{}}Bug\\Localization \\ Technique\end{tabular} & \begin{tabular}[c]{@{}c@{}}Top 1\\ \%\end{tabular} & 
%			\begin{tabular}[c]{@{}c@{}}Top 5\\ \%\end{tabular} & 
%			\begin{tabular}[c]{@{}c@{}}Top 10\\ \%\end{tabular} &
%			\begin{tabular}[c]{@{}c@{}} MRR \end{tabular} & 
%			\begin{tabular}[c]{@{}c@{}} MAP \end{tabular} \\
				{$\alpha$} 
				 & {$\beta$}
				 	 & $\gamma$
				  & \textbf{Top 1} \% 
				  & \textbf{Top 5} \% & 
			\textbf{Top 10} \% &
			 \textbf{MRR}  & 
			 \textbf{MAP}  \\
			 \hline
			{0.2} 
			& 0.2 &0.6                                                                                                                                               & 16.13                                               & 40.40                                            & 52.99                                                &   0.26  & 0.25    \\ \hline
			{0.3} 
			& 0.2  &0.5                                                                    & 18.26                                               & 42.83                                              & 55.73                                             &   0.29  &   0.27  \\ 
			\hline
			
			{0.4}  & 0.2
			& 0.4
			&20.40 &
			45.32 &
			57.28 & 0.31 &
			0.29     \\  
			\hline
			{Average}      
			&                                                                              &                                           & 18.26\%                                                 & 42.85\%                                                 & 55.33\%                                                  &     0.29 &  0.27    \\ 
			\hline
\end{tabular}}
\centering
\end{table}
\begin{figure}
	\centering
	\includegraphics[scale=0.80]{Performance-rvsm+simi+co}
	\caption{The impact of $\alpha$ on bug localization performance (Top-1, Top-5, Top-10)} for proposed approach1.
	\label{fig:PerformanceTopK}
\end{figure}

\textbf{Weighting Function for VSM+Co-ocuurence Ranking:}
For our approach2 we also compute performance TopK accuracy, MRR and MAP for different weighting function such as  \textit{ALPHA} is 0.2, 0.3, 0.4. The results are presented in Table \ref{tab:alphaApproach2}. Here, it shows, more $\alpha$ produces better performance. That means if we increase the co-occurence scores with higher weighting function, the better performance is resulted in this proposed approach.   We also illustrate the impact of $\alpha$ for Top-1, Top-5 and Top-10 retrieval on Eclipse dataset for approach 2 (VSM+Co-ocuurence Ranking) in Figure \ref{fig:PerformanceTopK2}.
\begin{table}[htbp]
	\centering
		\caption{Performance of (VSM+Co-Occerence) for different weighting factors}
		\label{tab:alphaApproach2}
		\resizebox{3.4in}{!}{%
			\begin{tabular}{c|l|c|c|c|c|c}
				\hline
				%\begin{tabular}[c]{@{}c@{}}\#Bugs for \\ developing \\ map databases\end{tabular} &
				{$\alpha$} 
		
			& \textbf{Top 1} \% 
			& \textbf{Top 5} \% & 
			\textbf{Top 10} \% &
			\textbf{MRR}  & 
			\textbf{MAP}  \\
			\hline
				{0.2} 
				                                                                                                                                           & 26.72                                            & 51.94                                            & 62.43                                                &   0.37  & 0.36    \\ \hline
				{0.3} 
			                                                                 & 28.06                                               & 53.35                                              & 63.24                                             &   0.39  &   0.37  \\ 
				\hline
				{0.4} 
			
				&28.43 &
				53.86 &
				64.05 & 0.39 &
				0.37     \\  
				\hline
				{Average}      
				                                                                                                                        & 27.74\%                                                 & 53.05\%                                                 & 63.24\%                                                  &   0.38  &  0.37    \\ 
				\hline
		\end{tabular}}
	\centering
\end{table}
\begin{figure}
	\centering
	\includegraphics[scale=0.80]{Performance-vsm+co}
	\caption{The impact of $\alpha$ on bug localization performance (Top-1, Top-5, Top-10)} for proposed approach2.
	\label{fig:PerformanceTopK2}
	\end{figure}

\begin{figure}
	\centering
	\includegraphics[scale=0.80]{MRR-SWT-Zxing}
	\caption{The impact of $\alpha$ on bug localization performance (MRR)}
	\label{fig:MRR}
\end{figure}

\begin{figure}
	\centering
	\includegraphics[scale=0.80]{MAP-SWT-Zxing}
	\caption{The impact of $\alpha$ on bug localization performance (MAP)}
	\label{fig:MAP}
\end{figure}
\textbf{Impact of varying the value of alpha on BLuAMIR in terms of MAP and MRR}
We also evaluate the impact of co-occurence score
on bug localization performance, with different $\aleph$ values in terms of MAP and MRR for SWT and Zxing. At the beginning, the bug localization performance increases when the $\alpha$ value increases. However, after a certain point, further increase of the b value will decrease the performance. For example, Figure \ref{fig:MRR} and \ref{fig:MAP} show the bug localization performance (measured in terms of MRR and MAP) for the SWT and Zxing projects. When the $\aleph$ value increases from 0.1 to 0.4, both MRR and MAP values increases. Increasing $\alpha$ value further from 0.4 to 0.7 however leads to lower performance. Note that we obtain the best bug localization performance when $\alpha$ is between 0.3 and 0.4.

\subsection{\textbf{Answering Research Questions}}
\textbf{Answering RQ1:} To answer our \textit{RQ1}, we can go to Table \ref{tab:Performance2}, 64.05\% bugs are successfully located in Top-10 for Eclipse dataset. Our proposed tool BLuAMIR also outperforms on Zxing dataset for Top-10 retrieval, which is 75.00\%. However, on the subject system SWT, the Top-10 result of BLuAMIR is comparable, which is 80.61\%.

\textbf{Answering RQ2:}
We combine co-occurence rank with rVSM and Simi scores in Approach1 and VSM score in Approach2. We apply both of these approaches on Eclipse dataset. The results are provided in Table \ref{tab:Performance1} and Table \ref{tab:Performance2}. Here, we can see that both cases BLuAMIR outperforms in terms of Top-1, Top-5, Top-10, MRR and MAP. As co-occurence score is based on the association map between keywords and source files, thus no direct matching of vocabulary is required. Adding co-occurence score increases the performance in both cases also indicate that the association map is helping in locating buggy files. Therefore, vocabulary miss-match problem is resolved here. 

\textbf{Answering RQ3:}
We investigate how vocabulary miss-match problem is eliminated in BLuAMIR. We perform a query wise ranking comparison for Approach2 on Eclipse, which is given in Figure. 

Include rank improvement here

\textbf{Answering RQ4:}
We compare the performance of BLuAMIR with state-of-the-art bug localization technique, BugLocator proposed by \citet{Jian} on two dataset i.e., SWT and Zxing. These results can be found in Table \ref{tab:performance3}. For SWT, we can see our tool BLuAMIR performs better for Top-1, Top-5, MRR and MAP. However, for Top-10 BLuAmir is comparable with Buglocator. On the other hand, for Zxing our tool BLuAMIR outperforms for top-5, top-10 and MAP. So we can say that our proposed tool BLuAMIR outperforms most cases and comparable for a few cases with state-of-the-art bug localization technique.
\section{Related Work}
BugLocator
BLuIR
BugCenter
Amalgum
\section{Threats To Validity}\label{sec:threats}


\section{Conclusion and Future Work}\label{summary}



\bibliographystyle{plainnat}
\scriptsize
\bibliography{test}
%\begin{thebibliography}{00}
%\bibitem{b1} G. Eason, B. Noble, and I. N. Sneddon, ``On certain integrals of Lipschitz-Hankel type involving products of Bessel functions,'' Phil. Trans. Roy. Soc. London, vol. A247, pp. 529--551, April 1955.
%\bibitem{b2} J. Clerk Maxwell, A Treatise on Electricity and Magnetism, 3rd ed., vol. 2. Oxford: Clarendon, 1892, pp.68--73.
%\bibitem{b3} I. S. Jacobs and C. P. Bean, ``Fine particles, thin films and exchange anisotropy,'' in Magnetism, vol. III, G. T. Rado and H. Suhl, Eds. New York: Academic, 1963, pp. 271--350.
%\bibitem{b4} K. Elissa, ``Title of paper if known,'' unpublished.
%\bibitem{b5} R. Nicole, ``Title of paper with only first word capitalized,'' J. Name Stand. Abbrev., in press.
%\bibitem{b6} Y. Yorozu, M. Hirano, K. Oka, and Y. Tagawa, ``Electron spectroscopy studies on magneto-optical media and plastic substrate interface,'' IEEE Transl. J. Magn. Japan, vol. 2, pp. 740--741, August 1987 [Digests 9th Annual Conf. Magnetics Japan, p. 301, 1982].
%\bibitem{b7} M. Young, The Technical Writer's Handbook. Mill Valley, CA: University Science, 1989.
%\end{thebibliography}



\end{document}
